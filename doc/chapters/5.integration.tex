\chapter{Integrating Accelerators}
\label{chap_integration}

The user logic (or the accelerator) should have an interface which has a subset of signals described in Table~\ref{tab:user_sigs}.
The user logic can have any number of sub modules, but the top most module name should be user\_logic\_top since it is the instantiation name used in SCAS.
If following command line implementation, three files are required for generating the final bitstream.\\
\begin{enumerate}
\item fpga\_spec.h~~~~~~~~~~~~~~~~~~~~~: The SCAS specification function
\item V6\_scas.tcl/V7\_scas.tcl ~~~~: Implementation TCL file
\item make\_fpga.sh~~~~~~~~~~~~~~~~~~~: Script to run the TCL file
\end{enumerate}
The specifications in fpga\_spec.h are given in Table~\ref{tab:spec}.
\input \TBLDIR/specs.tex
When enabling or disabling the Ethernet interface, a few things has to be noted.
For V6, the netlist for the Ethernet controller is installed in the global user library.
If the interface is disabled in the fpga\_spec.h file, the user has to set ENET\_ENABLE to 0 in the tcl file also.
This is to exclude the specific constraints used for Ethernet controller during implementation.
Otherwise this will cause errors in the ISE translate phase.
For V7, the Ethernet core has to be generated by the user with the name v7\_emac\_controller in SGMII standard.
This is since generating this core requires a special license from Xilinx and hence cannot be publically distributed.
The netlist for this core can be then stored in the global location or can be stored locally in the current working directory.

If the user logic is using sub modules, the names of these files have to be added in the tcl file using \emph{xfile add} attribute along with the presently listed files.
Additional constraints files can be also added in a similar manner.
Now execute the script make\_fpga with v6 or v7 as the argument depending upon the target board.
The script fetches all the required source, netlist and constraints files from the global location and also uses the local user logic files.
The final output will be top\_v6.bit or top\_v7.bit bitstream depending upon the target board.
Since the intermediate files share the same name, do not run the script concurrently for different boards from the same working directory.