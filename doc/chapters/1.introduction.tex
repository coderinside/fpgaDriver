\chapter{Introduction}
\label{chap_intro}

Reconfigurable computing (RC) aims to fill the gap between hardware and software to achieve much higher performance than software, while maintaining a higher level of flexibility than hardware.
In order to achieve this performance benefit while supporting wide varieties of applications, reconfigurable systems are usually formed with the combination of reconfigurable logic and general purpose processors (GPPs).
Presently the most widely used reconfigurable logic in RC is the Field Programmable Gate Arrays (FPGAs) due to their high logic capacity, flexible routing architecture and wider tool support.
The reconfigurable logic implements custom hardware accelerators, which along with the software executed on the GPP provide much higher system performance compared to complete software implementations.

Although RC has been widely adopted in custom computing systems, it has not found acceptance into every day computing and scientific research.
Most of the custom RC applications are developed as standalone systems with very limited portability and reusability since there is no unified software and hardware communication interface for them.
Developers have to design communication and reconfiguration infrastructure even before testing the functionality of the target application.
This means there is lower productivity and longer design and implementation time.

Scalable-Configurable AXI Switch (SCAS) is an attempt to encourage the adoption of FPGA based accelerators on commercial computers.
This platform enables developers to quickly integrate hardware accelerators to a reusable communication as well as reconfigurable infrastructure capable of very high performance throughput. 
It supports PCIe, DRAM and Ethernet interfaces along with a configurable number of AXI-stream based communication channels to the user logic.
In addition to this, the user logic is also provided an address/data (PIO) interface for both PCIe and DRAM.
It allows run-time configuration of the communication pathways along with support for run-time reconfiguration of the user logic.
The hardware as well as software infrastructure is made completely open-source so that developers can adapt the platform to cater their specific requirements.
The present version is fully portable across Xilinx Virtex-6 FPGA based ML605 and Virtex-7 FPGA based VC707.
To enable this portability, some performance benefits of VC707 are sacrificed.
A high-performance platform will be released in the future exploiting the full capabilities of VC707.